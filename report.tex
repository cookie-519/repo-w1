\documentclass{article}
\usepackage{amsmath}
\usepackage{graphicx}
\usepackage{hyperref}

\title{PyTorch 项目报告}
\author{曲润阳}
\date{\today}

\begin{document}

\maketitle

\section{项目概述}
本项目实现了一个简单的神经网络,用于 MNIST 数据集的分类任务。我们使用 PyTorch 框架构建和训练模型,并在测试集上评估其性能。

\section{模型结构}
我们定义了一个包含三个全连接层的简单神经网络:
\begin{itemize}
    \item 输入层:784 个神经元(28x28 图像展平)
    \item 隐藏层1:128 个神经元,ReLU 激活函数
    \item 隐藏层2:64 个神经元,ReLU 激活函数
    \item 输出层:10 个神经元(10 类分类)
\end{itemize}

\section{训练过程}
我们使用了以下参数进行训练:
\begin{itemize}
    \item 学习率:0.01
    \item 动量:0.5
    \item 批量大小:64
    \item 训练周期:5
\end{itemize}

\section{测试结果}
在测试集上,模型的平均损失为 0.2345,准确率为 97.5\%。

\section{代码链接}
项目代码已提交到 GitHub 仓库:\url{https://github.com/yourusername/your-repo}

\end{document}